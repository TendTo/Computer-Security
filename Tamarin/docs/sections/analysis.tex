\section{Analisi formale}

\subsection{Modello del protocollo}
Per realizzare un'analisi formale, il protocollo è stato scritto utilizzando la sintassi prevista da \gls{tamarin}.
Utilizzando la tecnica del \gls{multiset-rewriting}, è possibile specificare una serie di regole di riscrittura
che modellino il comportamento degli agenti che vi partecipano. \\

In questa analisi si procederà assumendo che le primitive crittografiche utilizzate dal protocollo siano sicure.
Nello specifico, non è possibile in nessuna circostanza ottenere un testo in chiaro a partire da un testo cifrato senza conoscere la chiave,
le firme digitali non sono falsificabili e le funzioni hash sono collision-resistant. \\

Sotto queste ipotesi, ci si concentrerà invece sullo scambio di messaggi attraverso la rete fra client e server.

\subsection{Modello di attaccante}
Il modello di attaccante adottato è quello classico Dolev-Yao \cite{art:dolev-yao}. \\
In sintesi, l'attaccante ha il completo controllo del canale di comunicazione.
È quindi in grado di intercettare qualsiasi messaggio e di conoscerne il contenuto (che può essere cifrato o non),
modificarlo a piacimento e inviarlo ad un destinatario arbitrario. \\
Le uniche limitazioni imposte all'avversario sono relative alle ipotesi fatte in precedenza
sulla bontà delle primitive crittografiche, che non possono essere violate. \\

Al fine di rendere più verosimile il modello, vi sono delle regole artificiali che rivelano informazioni aggiuntive all'attaccante.
Questo permette di valutare che abusi il protocollo è in grado di sopportare prima di non garantire più la proprietà di sicurezza richiesta.

\subsection{Proprietà di sicurezza}
Ogni sottosezione di \gls{mtproto} soddisfa uno specifico obiettivo.
Nel complesso, l'intero protocollo dovrebbe garantire le seguenti proprietà:
\begin{itemize}
    \item \textbf{Secrecy:} se un messaggio $m$ viene scambiato fra due agenti onesti $A$, $B$ durante una session $S$,
          il contenuto del messaggio è noto solo ad $A$ e $B$
    \item \textbf{Forward secrecy:} una chiave di sessione $s$ non viene compromessa
          anche se la chiave a lungo termine usata per generarla viene compromessa
    \item \textbf{Authentication:} se $B$ riceve un messaggio che, seguendo il protocollo,
          attribuisce ad $A$, questo è stato effettivamente inviato da $A$
    \item \textbf{Integrity:} se $A$ invia un messaggio $m$ a $B$
          e questo riceve un messaggio $m'$ da $A$, ne segue che $m' = m$
\end{itemize}

\subsection{Authorization key protocol}
La verifica del protocollo di creazione dell'Authorization Key dimostra in maniera formale le proprietà di sicurezza attese.

\subsubsection{Authentication}

\lstinputlisting[caption=Il client non è autenticato agli occhi del server.
Anche un avversario può iniziare legittimamente il protocollo. Questo lemma è falso.,
label=cod:lemma:client_auth,linerange={163-171}]{../mtproto/MTProto_AuthKey_Creation.spthy}

\lstinputlisting[caption={Se il client ha ricevuto i parametri relativi a DHE, è stato il server ad inviarli.},
label=cod:lemma:auth_dh,linerange={177-198}]{../mtproto/MTProto_AuthKey_Creation.spthy}

\lstinputlisting[caption={Se il protocollo si è concluso con successo, il client e il server hanno entrambi portato 
a termine il protollo},
label=cod:lemma:success_auth,linerange={204-224}]{../mtproto/MTProto_AuthKey_Creation.spthy}

\subsubsection{Secrecy e Forward Secrecy}

\lstinputlisting[caption={Se la chiave privata del server non è compromessa e non è rivelata per errore,
l'avversario non arriva a conoscere la nonce nk condivisa
fra C ed S.},
label=cod:lemma:secret_nk,linerange={235-254}]{../mtproto/MTProto_AuthKey_Creation.spthy}

\lstinputlisting[caption={Se il client e il server hanno ottenuto una chiave condivisa kas,
sono i soli a conoscerla.},
label=cod:lemma:secret_kas,linerange={260-279}]{../mtproto/MTProto_AuthKey_Creation.spthy}

\lstinputlisting[caption={Se il client e il server hanno ottenuto una chiave condivisa kas,
sono i soli a conoscerla.
Questa rimane sicura anche se avvengono dei leak a posteriori di informazioni segrete,
garantendo forward secrecy, purchè ciò avvenga dopo il quarto messaggio del protocollo.},
label=cod:lemma:secret_kas_leaks,linerange={287-308}]{../mtproto/MTProto_AuthKey_Creation.spthy}

\subsubsection{Integrity}

\lstinputlisting[caption={Se il client e il server hanno ottenuto una chiave condivisa kas nella 
stessa sessione, questa è uguale per entrambi},
label=cod:lemma:agreement_kas,linerange={317-335}]{../mtproto/MTProto_AuthKey_Creation.spthy}


\lstinputlisting[caption={Se il client e il server hanno ottenuto una stessa chiave condivisa kas,
è perchè stanno partecipendo alla medesima sessione},
label=cod:lemma:agreement_session,linerange={341-359}]{../mtproto/MTProto_AuthKey_Creation.spthy}

\subsection{Secret chat}
Il protocollo originale delle chat segrete prevede che i due client utilizzino la chiave di autenticazione a lungo termine (\autoref{sec:auth-key}) che condividono con il server
per poter comunicare fra di loro, con il server che ne assicura l'autenticazione. \\
Nell'ipotesi che la chiave condivisa non sia compromessa e che il server non sia colluso, il protocollo è banalmente sicuro. \\
Per rendere l'analisi utile ed interessante, i messaggi di entrambi i client non sono cifrati con le chiavi di autenticazione.
Al contrario, sono trasmessi nel canale in plaintext. \\
Nonostante ciò, la verifica del protocollo per la creazione di Secret Chat dimostra in maniera formale le proprietà di sicurezza attese. \\

\subsubsection{Authentication}

\lstinputlisting[caption={Se entrambi gli utenti hanno confermato tramite l'applicazione di 
vedere la stessa chiave, allora il primo ha effettivamente ricevuto un messaggio dal secondo.},
label=cod:lemma:auth_r,linerange={121-135}]{../mtproto/MTProto_Secret_Chat.spthy}

\lstinputlisting[caption={Se entrambi gli utenti hanno confermato tramite l'applicazione di 
vedere la stessa chiave, allora il secondo ha effettivamente ricevuto un messaggio dal primo.},
label=cod:lemma:auth_i,linerange={141-155}]{../mtproto/MTProto_Secret_Chat.spthy}


\subsubsection{Secrecy}

\lstinputlisting[caption={Se entrambi gli utenti hanno confermato tramite l'applicazione di 
vedere la stessa chiave, allora questa è nota solo a loro.},
label=cod:lemma:secret_kir,linerange={164-177}]{../mtproto/MTProto_Secret_Chat.spthy}

\lstinputlisting[caption={Se entrambi gli utenti hanno confermato tramite l'applicazione di 
vedere la stessa chiave, l'avversario non è in grado di decifrare messaggi.},
label=cod:lemma:secret_m,linerange={183-193}]{../mtproto/MTProto_Secret_Chat.spthy}

\lstinputlisting[caption={Se gli utenti non hanno confermato tramite l'applicazione di 
vedere la stessa chiave, un attaccante potrebbe aver impersonato uno degli agenti. Questo lemma è falso.},
label=cod:lemma:secret_m_no_check,linerange={200-210}]{../mtproto/MTProto_Secret_Chat.spthy}
