\section{Analisi formale}

\subsection{Modello del protocollo}
Il modello del protocollo, realizzato per consentire un'analisi formale dello stesso, è realizzata con \gls{tamarin}.
Utilizzando la tecnica del \gls{multiset-rewriting}, è possibile specificare una serie di regole di riscrittura
che modellino fedelmente il comportamento degli agenti che vi partecipano. \\

In questa analisi si procederà assumendo che le primitive crittografiche utilizzate dal protocollo siano sicure.
Nello specifico, non è possibile in nessuna circostanza ottenere un testo in chiaro a partire da un testo cifrato senza conoscere la chiave, 
le firme digitali non sono falsificabili e le funzioni hash sono collision-resistant. \\

Sotto queste ipotesi, ci si concentrerà sullo scambio di messaggi attraverso la rete da parte di client e server.

\subsection{Modello di attaccante}
Il modello di attaccante adottato è quello classico Dolev-Yao \cite{art:dolev-yao}. \\
In sintesi, l'attaccante ha il completo controllo del canale di comunicazione. 
È quindi in grado di intercettare qualsiasi messaggio e di conoscerne il contenuto (che può essere cifrato o non),
modificarlo a piacimento e inviarlo ad un destinatario arbitrario. \\
Le uniche limitazioni imposte all'avversario sono relative alle ipotesi fatte in precedenza 
sulla bontà delle primitive crittografiche, che non possono essere violate. \\

Al fine di rendere più verosimile il modello, alcuni dei server coinvolti nel protocollo potrebbero essere compromessi,
rivelando informazioni aggiuntive all'attaccante.

\subsection{Proprietà di sicurezza}
Ogni sottosezione di \gls{mtproto} soddisfa uno specifico obiettivo.
Nel complesso, l'intero protocollo dovrebbe garantire le seguenti proprietà:
\begin{itemize}
    \item \textbf{Secrecy:} se un messaggio $m$ viene scambiato fra due agenti onesti $A$, $B$ durante una session $S$, 
    il contenuto del messaggio è noto solo ad $A$ e $B$
    \item \textbf{Forward secrecy:} una chiave di sessione $s$ non viene compromessa 
    anche se la chiave a lungo termine usata per generarla viene compromessa
    \item \textbf{Authentication:} se $B$ riceve un messaggio che, seguendo il protocollo,
    attribuisce ad $A$, questo è stato effettivamente inviato da $A$
    \item \textbf{Integrity:} se $A$ invia un messaggio $m$ a $B$ 
    e questo riceve un messaggio $m'$ da $A$, ne segue che $m' = m$
\end{itemize}