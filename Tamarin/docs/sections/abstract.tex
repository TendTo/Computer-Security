\begin{abstract}
    Nell'ambito della sicurezza informatica avere delle certezze sulla bontà di un protocollo è un obiettivo ambito.
    Non è sempre facile riuscire a dimostrare le proprietà desiderate in maniera soddisfacente e soprattutto rigorosa.
    Per quanto l'esaminatore possa essere attento ed esperto, c'è sempre la possibilità che un vettore d'attacco non sia individuato correttamente,
    o, al contrario, non si riesce ad avere la certezza di avere fra le mani un protocollo effettivamente sicuro. \\

    Per questo motivo esistono tool come \gls{tamarin}, in grado di effettuare un'analisi formale assistita di protocolli di sicurezza.
    Strumenti di questo genere sono spesso impiegati in ambiti dove ottenere delle garanzie verificabili sulle proprietà
    di sicurezza fornite da un protocollo è di fondamentale importanza per il business. \\

    In questa relazione verrà sottoposto a un analisi formale MTProto v2.0, 
    la seconda (e attuale) versione del protocollo di comunicazione ad hoc utilizzato dall'app di messaggistica istantanea Telegram.
\end{abstract}
