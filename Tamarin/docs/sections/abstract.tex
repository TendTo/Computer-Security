\begin{abstract}
    Nell'ambito della sicurezza informatica avere delle certezze sulla bontà di un protocollo è un obiettivo ambito.
    Tuttavia non è sempre facile riuscire a dimostrare le proprietà desiderate in maniera soddisfacente e soprattutto formale.
    Per quanto l'esaminatore possa essere attento ed esperto, c'è sempre la possibilità che un vettore d'attacco non sia individuato correttamente,
    o, al contrario, non si riesce ad essere sicuri di avere fra le mani un protocollo effettivamente sicuro. \\

    Per questo motivo esistono tool come \gls{tamarin}, in grado di effettuare un'analisi formale assistita di protocolli di sicurezza.
    Strumenti di questo genere sono spesso impiegati in ambiti dove ottenere delle garanzie universalmente verificabili sulle proprietà
    garantite da un protocollo è di fondamentale importanza per il business. \\

    In questa relazione verrà sottoposto a questo tipo di analisi MTProto 2.0, 
    un protocollo di comunicazione ad hoc utilizzato dall'app di messaggistica istantanea Telegram.
\end{abstract}
