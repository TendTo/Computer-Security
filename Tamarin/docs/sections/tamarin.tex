\section{Tamarin prover}

\subsection{Input}

In questa introduzione ci si soffermerà di una breve descrizione sul funzionamento di \gls{tamarin}. \\
\gls{tamarin} è uno strumento in grado di effettuare la modellazione simbolica e, a partire da questa, l'analisi formale di protocolli di sicurezza.
Per poter avviare una dimostrazione, bisogna fornire in input
\begin{itemize}
    \item il modello che rappresenta fedelmente e correttamente il protocollo
    \item le libertà concesse all'avversario
    \item le proprietà di sicurezza che si desidera assicurare
\end{itemize}
Quello che il tool cercherà di fare è dimostrare se le proprietà indicate sono verificate dal protocollo o, in caso contrario, un attacco che le comprometta. \\

\subsection{Multiset rewriting}
La tecnica utilizzata da \gls{tamarin} per modellare e poi analizzare un protocollo è quella del \gls{multiset-rewriting}, che si ottiene specificando un insieme di regole di riscrittura. \\

Lo stato corrente viene alterato in maniera coerente alle regole imposte, che ne definiscono le modalità di transazione.
Consultando è possibile stabilire, ad esempio, la conoscenza dell'avversario ad un istante $t$ o i messaggi pubblicati sul canale.

\gls{tamarin} è in grado di applicare queste regole in maniera autonoma, al fine di produrre una prova di correttezza o un controesempio.
Tuttavia, poiché si tratta di un problema indecidibile, la risoluzione automatica potrebbe non trovare un risultato.
In questi casi è prevista una modalità interattiva che permette all'utente di analizzare lo stato corrente per guidare il programma nella risoluzione.