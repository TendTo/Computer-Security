\section{Tamarin prover}

\subsection{Input}

In questa introduzione ci si soffermerà di una breve descrizione sul funzionamento di \gls{tamarin}. \\
\gls{tamarin} è uno strumento in grado di effettuare la modellazione simbolica e, a partire da questa, l'analisi formale di protocolli di sicurezza.
Per poter avviare una dimostrazione, bisogna fornire in input
\begin{itemize}
    \item il modello che rappresenta fedelmente e correttamente il protocollo
    \item le libertà concesse all'avversario
    \item le proprietà di sicurezza che si desidera assicurare
\end{itemize}
Quello che il tool cercherà di fare è dimostrare se le proprietà indicate sono verificate dal protocollo o, in caso contrario, un attacco che le comprometta.

\subsection{Multiset rewriting}
La tecnica utilizzata da \gls{tamarin} per modellare e poi analizzare un protocollo è quella del \gls{multiset-rewriting}, che si ottiene specificando un insieme di regole di riscrittura. \\

Lo stato corrente viene alterato in maniera coerente alle regole imposte, che ne definiscono le modalità di transazione.
Consultandolo è possibile stabilire, ad esempio, la conoscenza dell'avversario ad un istante $t$ o i messaggi pubblicati sul canale. \\
All'inizio, lo stato iniziale è il multiset vuoto, che si popola man mano secondo le regole di transizione.

\gls{tamarin} è in grado di applicare queste regole in maniera autonoma, al fine di produrre una prova di correttezza o un controesempio.
Tuttavia, poiché si tratta di un problema indecidibile, la risoluzione automatica potrebbe non trovare un risultato.
In questi casi è prevista una modalità interattiva che permette all'utente di analizzare lo stato corrente per guidare il programma nella risoluzione.

\subsection{Rule}
Le regole di riscrittura si compongono di tre parti:
\begin{itemize}
    \item \textbf{le precondizioni}, indicano i fatti che devono essere presenti nello stato corrente del programma per far si che la regola sia applicabile. \\
          A meno che non sia indicata con il simbolo \textit{!}, viene rimossa dallo stato dopo essere stata utilizzata
    \item \textbf{le azioni}, vengono salvate come traccia. Consultando quest'ultima attraverso i lemmi, è possibile verificare le proprietà di sicurezza
    \item \textbf{le conclusioni}, vengono aggiunte allo stato corrente del programma, sostituendo gli input
\end{itemize}

Più formalmente, dato uno stato $S$ le regole con forma $l \rightarrow [a] \rightarrow r$, una regola di transizione applicabile, 
per la quale tutte le premesse che la caratterizzano sono già nello stato corrente del programma, 
fà si che $S \xrightarrow{\text{diventa}} (S \setminus l ) \cup r \Rightarrow l \subseteq S$.

\begin{lstlisting}[caption=Regole che modellano l'invio e ricezione di un messaggio cifrato. Si assuma che le premesse \textit{Pk(pk)} e \textit{Sk(sk)} siano già presenti nello stato.,
    label=cod:rule]
rule send_cryptotext:
    [ Fr(~nonce), !Pk(pk) ] // Premesse
    --[ SendCrypt(~nonce) ]-> // Azione
    [ Out(senc{~nonce}pk) ] // Conclusione

rule send_cryptotext:
    [ !Sk(sk), In(senc{m}pk(sk)) ] // Premesse
    --[ ReceiveCrypto(m) ]-> // Azione
    [] // Conclusione
\end{lstlisting}

Nell'esempio riportato (\autoref{cod:rule}), la prima regola viene applicata se nello stato è presente il fatto \textit{Pk(pk)} che,
poiché indicata con il simbolo \textit{!}, non viene rimossa. \\
Viene anche introdotta una variabile mai usata prima \textit{~nonce} tramite il termine speciale \textit{Fr}.
Si aggiunge allo stato la conclusione \textit{Out(senc\{~nonce\}pk)}, cioè la nonce cifrata.
La regola registra sulla traccia l'azione \textit{SendCrypt(~nonce)}. \\
Successivamente è possibile applicare la regola \textit{send\_cryptotext}.
Con la premessa \textit{Sk(sk)} si ha accesso alla chiave segreta in grado di decifrare il messaggio ricevuto dal canale,
che viene poi registrato sulla traccia.

\subsection{Lemma}
I lemma sono delle affermazioni che modellano le proprietà che si vuole dimostrare il protocollo possieda.

\begin{lstlisting}[caption=Lemma che modella la proprietà di segretezza del messaggio.,
    label=cod:lemma]
lemma secrecy:
    "
    not ( Ex m #i #j .
            ReceiveCrypto(m) @ i
            & K(m) @ j
        )
    "
\end{lstlisting}

Il lemma di esempio (\autoref{cod:lemma}) può essere letto come:
\begin{quote}
    Non esiste un messaggio $m$, e istanti \textit{\#i} e \textit{\#j}, tali che
    si verifica l'evento \textit{ReceiveCrypto(m)} nell'istante  \textit{\#i}
    e l'attaccante conosce il valore di $m$ nell'stante  \textit{\#j}.
\end{quote}
