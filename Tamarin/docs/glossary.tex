\newglossaryentry{tamarin}
{
    name={Tamarin},
    first={Tamarin Prover},
    text={Tamarin},
    description={
            Tamarin prover è un tool in grado di verificare la correttezza formale di protocolli di sicurezza.
            both falsification and unbounded verification in the symbolic model.
        }
}
\newglossaryentry{cia}
{
    name={Confidentiality, Integrity, Availability (CIA)},
    first={Confidentiality, Integrity, Availability (CIA)},
    text={CIA},
    description={
            Confidentiality, Integrity e Availability sono le proprietà di sicurezza base più comuni che ci si aspetta un protocollo garantisca.
            \begin{itemize}
                \item Confidentiality: il protocollo è confidenziale, ovvero non è possibile conoscere il contenuto di un messaggio.
                \item Integrity: il messaggio non è danneggiato o alterato.
                \item Availability: il messaggio o il servizio è disponibile.
            \end{itemize}
        }
}
\newglossaryentry{multiset-rewriting}
{
    name={Multiset rewriting},
    first={multiset rewriting},
    text={multiset rewriting},
    description={
            Tecnica utilizzata da \gls{tamarin} per modellare e poi analizzare un protocollo.
            Con questo formalismo si è in grado di passare da uno stato ad un altro del sistema attraverso le regole definite.
            Si è quindi in grado di rappresentare sia il comportamento degli agenti che partecipano al protocollo, sia dell'avversario.
        }
}
\newglossaryentry{mtproto}
{
    name={MTProto},
    first={Mobile Transport Protocol (MTProto)},
    text={MTProto},
    description={
            Protocollo sviluppato da Telegram per la comunicazione sicura tra client e server.
            Viene rivolto in particolare ai dispositivi mobili.
        }
}
